\subsection{Movimientos}

Para generar la lista de los movimientos que puede realizar un jugador en un
estado dado del juego se han de recorrer todas las fichas que posee, probar
todas las rotaciones posibles en las posiciones válidas y quedarse con las que
cumplen las reglas de pisar niguna pieza ya puesta y no tocar ninguna arista del
mismo color.

Esta tarea se siplifica enormemente con las tablas de esquinas [ver
\ref{subsec:piezas}]. Para obtener una lista de las posiciones candidatas dónde
situar una nueva pieza simplemente hay que recolectar la lista de esquinas
exteriores de cada pieza colocada en el tablero y eliminar las duplicadas. Una
vez se tenga esta lista, se recorren las piezas no jugadas y, por cada rotación
posible, se intentan colocar en colocar en todas las esquinas exteriores, el
conjuto de las que puedan ser colocadas representa los movimientos posibles del
jugador.

A continuación se muestra el pseudocódigo de este algoritmo:

\begin{algorithm}
\caption{get-movements}\label{get-movements}
\begin{algorithmic}
	\Function{get-movements}{$node, player$}
	\For{$pos\ in\ $valid-external-corners$(node)$}
		\For{$tile\ in\ $held-tiles$(player)$}
			\For{$rot\ in\ $possible-rotations$(tile)$}
				\For{$icorner\ in\ $internal-corners$(tile)$}
					\State $movement \gets\ $gen-movement$(tile, rot, icorner, pos)$
					\If{valid-movement$(movement)$}
						\State $movements \gets [movements, movement]$
					\EndIf
				\EndFor
			\EndFor
		\EndFor
	\EndFor
	\State \Return $[movements, pass]$ \Comment{Last movement always is pass}
	\EndFunction
\end{algorithmic}
\end{algorithm}
