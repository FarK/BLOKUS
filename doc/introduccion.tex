En este trabajo se ha realizado una implementación en LISP del juego de mesa
\textbf{Blokus} con el fin de probar y experimentar con el algoritmo
\textbf{minimax} para juegos de múltiples jugadores.

\subsection{El juego}
\textsl{Blokus} es un juego de mesa para 4 jugadores (aunque hay versiones con
más y mejos contrincantes) en el que cada jugador coloca por turnos en un
tablero de $20\times20$, una de las 21 piezas con las que inicia la partida.
Cada pieza puntúa según el número de casillas que ocupe y su colocación se rige
por las siguientes normas:

\begin{itemize}
	\item No pueden solaparse
	\item Sus ejes no pueden estar adyacentes a los de una pieza del mismo
		color
	\item Al menos un vértice debe ser adyacente a una pieza del mismo color
	\begin{itemize}
		\item En el primer turno, la pieza colocada deberá ocupar la
			esquina del tablero que le corresponda al jugador
	\end{itemize}
\end{itemize}

El juego se termina cuando:
\begin{itemize}
	\item Un jugador coloca todas sus piezas
	\item Ningún jugador puede colocar ninguna pieza
\end{itemize}

\subsection{El programa}
\label{subsec:el-programa}
El programa nos permite seleccionar el tipo de cada uno de los 4 jugadores:
\begin{itemize}
	\item \textbf{Ninguno}: Para indicar la ausencia del jugador
	\item \textbf{Humano}: Jugará el usuario mediante el teclado
	\item \textbf{Egoísta}: Jugará la máquina con una estrategia
		\textsl{egoísta}, ie: predecirá los movimientos suponiendo que
		tanto él como el resto de jugadores eligen la mejor jugada para
		sí mismos.
	\item \textbf{Paranoico}: Jugará la máquina con una estrategia
		\textsl{paranoica}, ie: predecirá los movimientos suponiendo que
		los demás jugadores eligen la peor jugada para él y él la mejor
		para sí mismo.
	\item \textbf{Aleatorio}: Elige una jugada aleatoria cada vez.
\end{itemize}

Además nos permitirá seleccionar el tiempo máximo que podrá ``\textsl{pensar}''
la máquina y/o la profundidad máxima del algoritmo minimax.
